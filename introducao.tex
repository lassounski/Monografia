\chapter{Introdução}

\section{Pré-introdução}

Programar é resolver um problema bem definido. Desenvolvimento de software é
definir o problema em primeiro lugar. Uma vez que você tenha a questão,
geralmente é fácil resolve-la, uma vez que você tenha as ferramentas e
conhecimento para usa-las. Se você não sabe a questão. as coisas ficam bem mais
difíceis. \cite{ProgrammingIsEasySoftwareDevelopmentIsHard}

\section{Apresentação}

O desenvolvimento de software passou, e de certa forma ainda passa, por uma fase
chamada de \emph{crise do software} \cite{HumbleProgrammer}. A Engenharia de
Software surgiu \cite{NaurRandell} numa tentativa de contornar esta crise, no
entanto, foi baseada na consideração extremamente equivocada de que o
desenvolvimento de software é um trabalho executado por trabalhadores manuais e
não um trabalhadores do conhecimento \cite[38]{XPTeles}. Isso fez com que a
Engenharia de Software falhasse na sua tentativa de contornar tal crise.

O desenvolvimento ágil de software, que neste ano de 2011 completa 10 anos,
surgiu \cite{AgileManifesto} para resolver esta crise que a Engenharia de
Software tradicional não conseguiu. De acordo com \cite{PMNetworkFailureDrop}, o
\textit{Chaos Manifesto 2011}\footnote{O \textit{Chaos Manifesto} é uma pesquisa
bienal realizada pelo \textit{The Standish Group} e teve início em 1994. As
pesquisas publicadas em um ano representam os dados do ano anterior.} mostra que
os resultados de 2010 representam, desde sua primeira edição, a maior taxa de
sucesso, que aumentou de 32\% em 2008 para 37\% em 2010. Segundo
\cite{ResumoChaosReport}, o \textit{The Standish Group} conclui que uma das
principais razões para o aumento da taxa de sucesso foi a utilização das
métodologias ágeis, que cresce a uma taxa de 22\%
CAGR\footnote{\href{http://en.wikipedia.org/wiki/Compound_annual_growth_rate}
{Compound annual growth rate}} e hoje são adotados em 9\% de todos os projetos
de TI em andamento e em 29\% dos novos projetos.

Diversos métodos e técnicas foram e continuam sendo desenvolvidos tendo como
base os princípios e valores ágeis \cite{BDDRodrigo}.
