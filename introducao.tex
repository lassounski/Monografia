\chapter{Introdução}

\section{Pré-introdução}

A tecnologia se desenvolve muito rapidamente e com isso, a maneira como se
desenvolve software também. O barateamento e o poder dos novos hardwares
permitem que o uso/criação de ferramentas e técnicas antes bloqueadas pela
limitação de memória e processamento.

Programar é resolver um problema bem definido. Desenvolvimento de software é
definir o problema em primeiro lugar. Uma vez que você tenha a questão,
geralmente é fácil resolve-lo, uma vez que você tenha as ferramentas e
conhecimento para usa-las. Se você não sabe a questão. as coisas ficam bem mais
difíceis. \cite{ProgrammingIsEasySoftwareDevelopmentIsHard}

\section{Apresentação}

É onde se faz a apresentação do problema, ressaltando os motivos mais
importantes que levaram à abordagem do tema.

O tema ou problema de pesquisa deve responder a o que vai ser investigado. A
escolha do tema não é criação individual do aluno, mas é feita com base em obras
que o bordem, trabalhadas por outros autores. A perspectiva adotada deve,
contudo, ser diferente, a partir da consulta à documentação para a realização do
projeto.

O tema escolhido deve ser tratado como um problema a ser resolvido. O
desenvolvimento do trabalho visa demonstrar uma posição única a respeito do tema
problematizado. Trata-se de definir os aspectos de dificuldade e de se
esclarecer os limites dentro dos quais a pesquisa e o raciocínio se
desenvolverão.

Para decidir sobre o tema, algumas questões precisam ser respondidas, tais como:

\begin{itemize}
    \item O tema é de interesse científico?
    \item É um assunto a ser provado ou resolvido?
    \item É um assunto que pode ser investigado?
    \item Há a disponibilidade de material bibliográfico sobre o assunto?
    \item Estou familiarizado com o tema?
    \item A pesquisa é viável, em termos de tempo e recursos disponíveis?
\end{itemize}

O problema, por sua vez, é manifestação da vontade de investigar o tema,
consistindo-se em uma questão não resolvida

Exemplo: A adoção da técnica XYZ de avaliação pode corrigir os problemas com a
evolução no aprendizado do aluno?


\section{Objetivos}

O aluno expõe os objetivos que o trabalho visa atingir, relacionados com a
contribuição que pretende trazer. Os objetivos visam responder para que o
trabalho será realizado. Os objetivos devem ser formulados com verbos no
infinitivo.


\subsection{Objetivo geral}

O Objetivo Geral expõe a razão maior do trabalho, ou o que se pretende com a
realização do trabalho. A formulação do objetivo geral usa verbos que admitem
interpretações amplas, como com provar, desenvolver, entender, conhecer e
aperfeiçoar.

Exemplo: O objetivo geral deste trabalho é estabelecer processos de apoio ao
desenvolvimento das aplicações Web, adaptados à natureza dessas aplicações e
tomando como base os aspectos observados na literatura sobre as características
dessas aplicações.


\subsection{Objetivos específicos}

Os Objetivos Específicos relacionam os resultados que se pretende alcançar, na
forma de etapas a serem cumpridas para a realização do objetivo geral. Os
objetivos específicos são o ponto de partida para a investigação, ordenados em
uma seqüência lógica de obtenção dos resultados desejados. Os verbos para
formular objetivos específicos devem admitir poucas interpretações, tais como
identificar, implementar, investigar, relacionar, escrever e aplicar.


\section{Justificativas}

As justificativas devem ser baseadas na relevância social e científica da
pesquisa proposta. Nesta etapa é respondido porque o trabalho será feito. A
justificativa deve deixar claro para o leitor:

\begin{itemize}
    \item O estágio de desenvolvimento em que o tema se encontra e a sua
          evolução histórica;
    \item O contexto em que o fenômeno ocorre;
    \item A importância social e científica da realização da pesquisa sobre o
          tema.
\end{itemize}

\section{Metodologia}

Aqui se anuncia, para o projeto, o tipo de pesquisa que será desenvolvida, bem
como os métodos e técnicas que serão adotados. Este item visa responder às
questões sobre como o trabalho será feito, com o que , onde e com quem será
realizado.

Exemplo: O trabalho será desenvolvido a partir de pesquisa bibliográfica e de
campo, através de método indutivo, utilizando levantamento de dados feito em
entrevistas com professores e coordenadores acadêmicos, ...

A monografia relata qual metodologia foi adotada, para contextualizar a pesquisa
realizada em uma base metodológica.


\section{Cronograma proposto}

A divisão da pesquisa em etapas de desenvolvimento, a partir dos objetivos
específicos, requer que seja estabelecido um tempo para a execução cronológica
dos trabalhos, dentro dos prazos estabelecidos para tal no calendário acadêmico.
A elaboração do cronograma responde à pergunta sobre quando cada fase do
trabalho terá lugar. As etapas do cronograma devem cobrir todo o período de
realização do trabalho demonografia, desde o refinamento do Projeto de
Monografia até a apresentação do trabalho monográfico à banca de avaliação. Cada
etapa a ser cumprida deve ser exibida como uma linha na tabela que compõe o
cronograma. As etapas são descritas após a exibição da tabela.
