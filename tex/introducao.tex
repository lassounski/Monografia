\chapter{Introdução}
A capacidade de visualizar as informações de maneira clara e organizada, é importante para qualquer usuário. Geralmente, as informações com as quais se lida no dia-a-dia são textuais e necessitam de tempo para serem interpretadas. Quando a quantidade destes dados é grande, se leva um tempo considerável para encontrar a informação de relevância. Nas décadas de 50, a informação começou a migrar do papel para os dispositivos eletrônicos, onde pode ser mais facilmente armazenada, conservada e organizada. O crescimento da quantidade de documentos armazenados tornou difícil a recuperação daqueles que são de interesse em um dado momento. A recuperação de informação é uma área da ciência da computação que lida com a armazenamento, organização, representação e acesso à itens de informação \cite{Yates1999}.

O tráfego na Internet tem crescido numa escala de 100\% a cada ano a partir da década de 90. Atualmente esta taxa diminuiu e está em 50\% \cite{Odlyzko2003}. A previsão é de que esta taxa diminua para 35\% no período até 2013 \cite{Cisco2009}. Apesar da diminuição da taxa de crescimento do tráfego na Internet, a quantidade de informações sendo lidas e armazenadas na grande rede cresce diariamente e a capacidade de extrair conhecimento relevante se torna uma dificuldade. Muitas vezes a informação está acessível, mas os métodos de busca atuais tornam difícil o seu acesso.

A crescente massa de informação criou a necessidade da concentração desta em um lugar em comum onde a informação possa ser visualizada, organizada e classificada. Existem diversos bancos de dados e bibliotecas que disponibilizam informação, seja ela científica ou tecnológica. Como exemplos pode-se destacar:
\begin{itemize}
    \item PubMed\footnote{\href{http://www.ncbi.nlm.nih.gov/pubmed}{http://www.ncbi.nlm.nih.gov/pubmed}}:
    possui mais de 21 milhões de citações de literatura biomédica do MEDLINE, periódicos científicos e e-books.
    \item Science Direct\footnote{\href{http://www.sciencedirect.com/}{http://www.sciencedirect.com/}}:
    banco de dados do SciVerse com mais de 10 milhões de artigos científicos e capítulos de livros.
    \item CiteULike\footnote{\href{http://www.citeulike.org/}{http://www.citeulike.org/}}:
    serviço gratuito para encontrar e gerenciar referências acadêmicas, possui atualmente cerca de 5 milhões de documentos.
    \item IEEE Xplore\footnote{\href{http://ieeexplore.ieee.org/Xplore/guesthome.jsp}{http://ieeexplore.ieee.org/Xplore/guesthome.jsp}}:
    possuindo 3 milhões de documentos, o portal da IEEE provê acesso à informação técnica em engenharia e tecnologia.
    \item ACM Digital Library\footnote{\href{http://dl.acm.org/}{http://dl.acm.org/}}:
    coleção de artigos e registros bibliográficos sobre os campos de computação e tecnologia da informação.
\end{itemize}

Independente do mecanismo de busca, se a base de dados for muito grande, o que acontece em muitos casos, serão retornadas grandes quantidades de documentos que tornam tedioso o processo de busca no conjunto de resultados. Quando esta quantidade de resultados ultrapassa centenas de itens, o processo de busca se torna cansativo mesmo fazendo uma leitura dinâmica nos resultados. Os sistemas de recomendação sugerem documentos relevantes baseado em informações providas por outros usuários de um sistema, gerando uma melhora no processo de busca \cite{Jannach2010}, mas este tipo de algoritmo não reflete na quantidade de documentos retornados. O PubMed utiliza um sistema de recomendação no campo "\emph{Related searches}", que disponibiliza frases de busca relacionadas com a submetida pelo usuário.

Para reduzir a quantidade de documentos disponibilizados em uma busca, pode-se dividi-los em categorias rotuladas por palavras-chave. Cada palavra-chave representa uma porção do total de documentos exibidos e representa um tema ou assunto abordado pelos documentos do sub-conjunto. A extração de palavras-chave é uma técnica que obtém termos descritores a partir dos textos de documentos, que podem ser usados para indexar/recuperar ou sumarizar os dados documentos.

Neste trabalho é apresentada uma abordagem de extração de palavras-chave a partir de documentos, e além disso é apresentada uma maneira de utilizar estas palavras para resumir sub-conjuntos destes documentos. Para avaliar os resultados obtidos foi criada uma biblioteca de construção de conjuntos de dados de teste obtidos do PubMed.

\section{Formulação do problema}
A grande massa de informação disponível criou a necessidade de mecanismos que possam organizá-la de maneira clara para os usuários. Esta dificuldade não é restrita à Internet, mas também existe em grandes empresas e instituições. Como uma grande parte desta informação está na forma de texto, muito esforço tem sido feito nas comunidades de recuperação de informação para melhorar os algoritmos de indexação/ordenação de documentos.

Como um exemplo pode-se destacar o \emph{National Center for Biotechnology Information} (NCBI), que provê informações científicas sobre biomedicina e genomas. É um  portal mantido pelo governo dos Estados Unidos da América utilizado por pesquisadores de todas partes do mundo servindo informações em seus estudos. Um de seus bancos de dados mais acessado é o PubMed, que guarda cerca de 21 milhões de artigos científicos na área de saúde, um número que cresce continuamente. Os resultados exibidos pelo PubMed formam um lista linear de artigos científicos que não possibilitam o usuário de ter um panorama dos tópicos sendo abordados por estes artigos ou de refinar os artigos até atingir uma necessidade em particular. Esta grande quantidade de informação gera dificuldade no processo de localização do material desejado com os métodos de busca tradicionais.

O processo de busca no portal se utiliza do modelo booleano\footnote{\href{http://www.nlm.nih.gov/pubs/techbull/ja97/ja97\_pubmed.html}{http://www.nlm.nih.gov/pubs/techbull/ja97/ja97\_pubmed.html}} onde o usuário especifica palavras-chave e as conecta utilizando operadores booleanos (\emph{AND,OR} e \emph{NOT}). As palavras-chave podem ser buscadas em diversos atributos de um documento como autor, ano de publicação, termos MeSH, resumo, título entre outros. 

Neste tipo de busca, os documentos retornados são filtrados de acordo com os termos que estão contidos nele e os que não estão contidos. Este tipo de busca é conhecido por retornar grandes conjuntos de resultados, além de possuir uma complicada lógica de formulação das buscas, logo um usuário inexperiente muitas vezes tem dificuldade em efetuar buscas satisfatórias \cite{Jackson2007}. 

Neste cenário, a quantidade de artigos retornados em uma busca geralmente é muito extensa e para diminuir o número de resultados o usuário se vê obrigado a inserir mais termos para refinar sua busca. Para exemplificar o problema, ao se fazer uma busca no PubMed com um termo de busca bem geral como \emph{“tuberculosis”}, foram retornados 192.499 artigos. Ao se especificar este termo para \emph{“human [Title/Abstract] AND mycobacterium tuberculosis [Mesh]”} a quantidade de resultados diminuiu para 3.084, mas ainda assim permanece grande para uma leitura rápida e o pesquisador se vê obrigado a especificar mais a sua busca. Como observado por \cite{Perez-Iratxeta2001}, em uma busca exploratória, onde o pesquisador seleciona termos de seu conhecimento para melhorar os resultados da busca, muitas vezes não se tem conhecimento de um determinado termo até que este seja identificado no texto.

\section{Objetivos}
Como objetivo principal, se propõe um método de refinamento de artigos do banco de dados PubMed através da utilização de palavras-chave. Cada palavra-chave representa um tema que é amplamente abordado por artigos retornados de uma busca ao banco. O método proposto permitirá o(a) pesquisador(a) encontrar com mais facilidade e rapidez artigos relevantes no PubMed. O objetivo secundário é a construção de uma \emph{interface} que melhore a navegação e visualização dos artigos.

Para atingir tal objetivo foi desenvolvido o sistema BioSearch Refinement, uma aplicação que pode sumarizar e dar uma visão geral dos artigos retornados em uma busca. Para destacar os temas sendo mais abordados em uma determinada busca são utilizadas palavras-chave. As palavras-chave podem ajudar o leitor a decidir se um determinado documento é de seu interesse ou não. Elas dão uma breve descrição sobre o conteúdo mais importante sendo abordado, mas muitas vezes os documentos não possuem palavras-chave e o processo de atribuição é muito custoso \cite{Lui2007}. Além da sumarização, palavras-chave podem ser usadas para indexar documentos e também para tornar uma busca mais precisa \cite{Turney1999}. Neste trabalho usamos as palavras-chave para fazer sumarização de grupos de documentos, onde cada palavra-chave representa uma parte deste grupo. Com estas palavras pode-se reduzir a quantidade de resultados retornados facilitando o processo de busca.

A ideia central do algoritmo é extrair  conceitos a partir dos resumos dos artigos, que serão utilizados para determinar quais artigos são referenciados por um determinado conceito. Os conceitos que referenciarem um maior número de artigos serão considerados mais generalistas e serão escolhidos como palavras-chave que sumarizam os principais assuntos de uma busca. A extração se utiliza de técnicas de processamento de linguagem natural e frequência de termos para determinar os conceitos. O algoritmo é aplicado no sistema BioSearch Refinement que é um sistema com o propósito de simplificar e ajudar o pesquisador a encontrar material relevante.

\section{Organização}
No capítulo 2 são apresentados os trabalhos relacionados e as metodologias usadas para a sumarização de documentos e extração de palavras-chave. Também são introduzidos os conceitos da recuperação de informação necessários para compreender o capítulo 3. No capítulo 3 é apresentado o sistema \emph{BioSearch Refinement} e as partes que o compõem, o \emph{Extraction Engine} e o \emph{PubMed Dataset}. O capítulo 4  apresenta os resultados obtidos para as partes que compõem o sistema e também discute estes resultados. No capítulo 5 são tiradas as conclusões sobre o trabalho feito como também são apresentadas as contribuições e possíveis trabalhos futuros.

