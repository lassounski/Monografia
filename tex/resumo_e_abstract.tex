\begin{resumo}
A quantidade de informação na \emph{Internet} está em constante crescimento, o que demanda técnicas modernas de recuperação de informação para organizá-la. O NCBI é um concentrador de informação biomédica e possui em sua estrutura o banco de dados PubMed que contém artigos científicos de jornais e revistas e outros recursos. O objetivo do \emph{BioSearch Refinement} é ajudar o pesquisador a encontrar informação relevante com mais facilidade e rapidez no banco de dados PubMed. O sistema permite a sumarização dos artigos através da representação de seus principais conceitos utilizando palavras-chave, e estas palavras podem ser usadas para refinar o conjunto de artigos exibidos. Neste trabalho também é apresentado o PubMed Dataset que provê o acesso ao NCBI e permite a criação de conjuntos de dados para teste com os dados do PubMed.
\end{resumo}

\begin{abstract}
The amount of information on the Internet is at constant growth and it demands modern information retrieval techniques to organize it. The NCBI is a biomedic information center, having in its structure the PubMed database that contains scientific articles and other resources. The BioSearch Refinement goal is to help researchers to find relevant information on PubMed easily and quickly. The system provides article summarization through the representation of its basic concepts using keywords, these keywords can be used to refine the set of articles displayed on the interface. This paper also presents the PubMed Dataset that provides access to NCBI and allows the creation of data sets using PubMed articles to test informaton retrieval systems.
\end{abstract}

