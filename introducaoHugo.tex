\chapter{Introdução}
O tráfego na internet têm crescido numa escala de 100\% a cada ano a partir da década de 90, atualmente esta taxa diminuiu e está em 50\% \cite{Odlyzko2003}. A previsão é de que esta taxa diminua para 35\% no período até 2013 (Cisco, 2009). Este aumento do tráfego por sua vez reflete na quantidade de informações sendo lidas e armazenadas na grande rede, o volume destas cresce a cada dia, e a capacidade de extrair conhecimento relevante se torna uma dificuldade. Muitas vezes a informação está acessível, mas os métodos de busca atuais tornam difícil o seu acesso.

A crescente massa de informação criou a necessidade da concentração desta em um lugar em comum onde a informação possa ser visualizada, organizada e classificada. Existem diversos bancos de dados e bibliotecas que disponibilizam informação, seja ela cientifica, cultural, empresarial ou tecnológica. Como exemplos podemos destacar:
\begin{itemize}
\item[PubMed] PubMed: Possui mais de 21 milhões de citações de literatura biomédica do MEDLINE, periódicos científicos e \emph{e-books}.
\item Science Direct: Banco de dados do SciVerse com mais de 10 milhões de artigos científicos e capítulos de livros. http://www.sciencedirect.com/
\item CiteULike: Serviço gratuito para encontrar e gerenciar referencias acadêmicas, possui atualmente cerca de 5 milhões de documentos. http://www.citeulike.org/
\item IEEE Xplore: Possuindo 3 milhões de documentos, o portal da IEEE provê acesso à informação técnica em engenharia e tecnologia. http://ieeexplore.ieee.org/Xplore/guesthome.jsp
\item ACM Digital Library: Coleção de artigos e registros bibliográficos sobre os campos de computação e tecnologia da informação. http://dl.acm.org/
\end{itemize}

Seja usada a busca booleana ou expressões regulares como mecanismo de busca, se a base de dados for muito grande, o que acontece em muitos casos, serão retornadas grandes quantidades de documentos que tornam o processo de busca tedioso. Os sistemas de recomendação que avaliam um determinado documento baseado na avaliação anterior de outros documentos similares, geram uma melhora nos resultados exibidos para um usuário, mas este tipo de filtragem reflete mais na ordenação dos documentos do que na quantidade. Uma maneira de ajudar o pesquisador é exibindo para ele palavras-chave.
As palavras-chave podem ajudar o leitor a decidir se um determinado documento é de seu interesse ou não, elas dão uma breve descrição sobre o conteúdo mais importante sendo abordado. Mas muitas vezes os documentos não possuem palavras-chave e o processo de atribuição é muito custoso (Lui, 2007). Além da sumarização, palavras-chave podem ser usadas para indexar documentos e também para tornar uma busca mais precisa (Turney, 1999). Neste trabalho usamos as palavras-chave para fazer sumarização de grupos de documentos, onde cada palavra-chave representa uma parte deste grupo.
Programar é resolver um problema bem definido. Desenvolvimento de software é
definir o problema em primeiro lugar. Uma vez que você tenha a questão,
geralmente é fácil resolve-la, uma vez que você tenha as ferramentas e
conhecimento para usa-las. Se você não sabe a questão. as coisas ficam bem mais
difíceis.
\cite{Doe:2009:Online}

\cite{ProgrammingIsEasySoftwareDevelopmentIsHard}

O desenvolvimento de software passou, e de certa forma ainda passa, por uma fase
chamada de \emph{crise do software} \cite{HumbleProgrammer}. A Engenharia de
Software surgiu \cite{NaurRandell} numa tentativa de contornar esta crise, no
entanto, foi baseada na consideração extremamente equivocada de que o
desenvolvimento de software é um trabalho executado por trabalhadores manuais e
não um trabalhadores do conhecimento \cite[38]{XPTeles}. Isso fez com que a
Engenharia de Software falhasse na sua tentativa de contornar tal crise.

O desenvolvimento ágil de software, que neste ano de 2011 completa 10 anos,
surgiu \cite{AgileManifesto} para resolver esta crise que a Engenharia de
Software tradicional não conseguiu. De acordo com \cite{PMNetworkFailureDrop}, o
\textit{Chaos Manifesto 2011}\footnote{O \textit{Chaos Manifesto} é uma pesquisa
bienal realizada pelo \textit{The Standish Group} e teve início em 1994. As
pesquisas publicadas em um ano representam os dados do ano anterior.} mostra que
os resultados de 2010 representam, desde sua primeira edição, a maior taxa de
sucesso, que aumentou de 32\% em 2008 para 37\% em 2010. Segundo
\cite{ResumoChaosReport}, o \textit{The Standish Group} conclui que uma das
principais razões para o aumento da taxa de sucesso foi a utilização das
métodologias ágeis, que cresce a uma taxa de 22\%
CAGR\footnote{\href{http://en.wikipedia.org/wiki/Compound_annual_growth_rate}
{Compound annual growth rate}} e hoje são adotados em 9\% de todos os projetos
de TI em andamento e em 29\% dos novos projetos.

Como o desenvolvimento ágil é relativamente novo, diversos métodos e técnicas
vem sendo desenvolvidos tendo como base os princípios e valores ágeis
\cite{BDDRodrigo}. Sendo técnicas emergentes, ainda são pouco discutidas no meio
acadêmico.

O objetivo desse trabalho é discutir estas técnicas, mostrando onde estas são
melhor aplicadas, comparando as diferentes abordagens para cada técnica, bem
como seus benefícios e malefícios.

